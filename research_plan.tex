\documentclass[10pt,letterpaper]{article}
\usepackage[utf8]{inputenc}
\usepackage{amsmath}
\usepackage{amsfonts}
\usepackage{amssymb}
\usepackage[margin=0.75in]{geometry}
\author{Lorne Arnold}
\title{Research Plan}
\begin{document}
\section*{Research Plan - Lorne Arnold}
\section{Purpose}
This research plan provides a framework for my planned research efforts over the next 5 years.
The plan is primarily intended for my own personal reference and prioritization .
However, I also intend to use it to communicate my research plans to others, as appropriate, and to get feedback on my plans for academic mentors.

\section{Research Topics and Goals}
My research interests broadly include statistical site characterization and probabilistic analysis, machine learning, rock mechanics and dynamics, numerical modeling using discrete element and finite element methods, and geotechnical instrumentation and exploration methods.

My research interests broadly include 1) natural hazard modeling and mitigation, and 2) geotechnical site characterization and analysis methods.
Within these areas, I have several sub-interests. 
Within natural hazard modeling and mitigation, my primary sub-interests are 1a) seismically-induced rock slope failure and 1b) the impact of climate change on geotechnical performance of infrastructure.
Within geotechnical site characterization and analysis, my primary sub-interests are 2a) developing open-source geotechnical analysis tools in Python and 2b) digitizing geotechnical data collection. 

\section{Funding Sources}
\begin{enumerate}
\item \textbf{Royalty Research Fund} (UW internal funding)\\ 
Status: Funded (June 2022)\\
Amount: \$38k\\
Duration: 1 year
\item \textbf{NSF Engineering Research Initiation (ERI)}\\
Status: Application in progress (due October 11, 2022)\\
Amount: \$200k\\
Duration: 2 years
\item \textbf{KEEN CURE}\\
Status: Tentative (pending discussion with Jeff W)\\
Amount: \$12k\\
Duration: 6 months
\item \textbf{NSF Faculty Early Career Development (CAREER)}\\
Status: First planned submission July 2024\\
Amount: \$500k\\
Duration: 5 years

\end{enumerate}
\subsection*{Other potential funding sources:}

\textbf{NSF RAPID grant:} These grants are awarded without panel review for collection of perishible data. Successful examples include instrumentation of rare planned events (e.g., rock blasting for a highway project). 
This grant could be a good candidate for collaboration with WSDOT by using a large infrastructure project activity as an opportunity to collect a research dataset.
\\
\\
\noindent \textbf{Unknown:} Some of my research interests involve tool development and data collection methods.
These are more applied and less fundamental than what the NSF typically funds, so I need to research alternative funding sources. 
WSDOT and the FHWA are the agencies I plan to look into first.

\section{Publication Plans}
I plan to publish two peer-reviewed papers per year (on average).
I would like at least one of those papers to be a journal article.
\\
\\
\noindent \textbf{Conference Papers:} My strategy for conference papers is to target publication at conferences I plan to attend in person and that will provide networking opportunities for funding and research collaboration.
My goal is to write papers of interest to the engineering community that show preliminary results in a limited way that allows the paper to be developed more fully for future journal publication.
\\
\\
\noindent \textbf{Journal Papers:} My strategy is to use feedback from my conference papers and new data from my research to develop more detailed papers for journal publication. 
I will target the highest impact journal that I think is a good fit for my paper, and pursue lower impact journals if I'm not able to get into my target journals.
\section{Student Involvement}
\section{Timeline}
\end{document}