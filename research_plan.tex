\documentclass[10pt,letterpaper]{article}
\usepackage[utf8]{inputenc}
\usepackage{amsmath}
\usepackage{amsfonts}
\usepackage{amssymb}
\usepackage{tabularx}
\usepackage[margin=0.75in]{geometry}
\usepackage{threeparttable}

\newcommand\C[1]

\author{Lorne Arnold}
\title{Research Plan}
\begin{document}
\footnotetext[1]{Updated \today}
\section*{Research Plan - Lorne Arnold}
\section{Purpose}
This research plan provides a framework for my planned research efforts over the next 5 years.
The plan is primarily intended for my own personal reference and prioritization.
However, I also intend to use it to communicate my research plans to others, as appropriate, and to get feedback on my plans from academic mentors.

\section{Research Topics and Goals}

My research interests broadly include 1) natural hazard modeling and mitigation, and 2) geotechnical site characterization and analysis methods.
Within these areas, I have several sub-interests:

\begin{enumerate}
\item Natural hazard modeling and mitigation
\begin{enumerate}
\item \textbf{Seismically-induced rock slope failure:} This interest is a continuation of my PhD research.
Through the dynamic Bonded Particle Model (BPM) I developed, I have several additional rock slope scenarios I can to run to produce a more comprehensive understanding of rock slope failure mechanisms.
I also plan to compare dynamic BPM to simplified analysis methods to create correlations that will allow for probabilistic and regional analysis.
\item \textbf{Climate impacts on geotechnical performance of soil slopes:} In several geologic settings, natural slopes are significantly steeper than any human-made unreinforced slope would be. 
However, these steep slopes are often very stable except for extreme loading conditions (e.g., earthquakes, extreme rainfall).
And most often, the types of failure they experience involve small, surficial slides.
Vegetation on slopes provides natural reinforcement in the plant root zone, and helps control the flow of water in the soil matrix and reduces the potential for hydraulically-induced instability.

As changes to our climate continue, several questions regarding steep slope stability warrant addressing:
How will changes in temperature extremes impact the vegetation providing stability in steep slopes? Which plants are most susceptible to negative impacts from climate change, and which are most critical to providing slope protection?
How will changes in temperature extremes impact the soil matrix? Will some physical processes involved in weathering accelerate, decelerate, or remain constant?

Answering these questions will require interdisciplinary study involving civil engineering, plant biology, and more.
\end{enumerate}
\item Geotechnical site characterization and analysis methods
\begin{enumerate}
\item \textbf{Developing open-source geotechnical analysis tools:} In geotechnical practice, most analysis is performed either in in-house spreadsheets or commercially-available software with limited scope of application.
As a consultant, I developed some code-based analysis tools, but I was expected to not publish these tools as open-source because that could damage the competitive advantage associated with an exclusive too.
Now in academia, I no longer have this constraint and am motivated to develop open-source tools to be used in everyday geotechnical work.
The benefits to the geotechnical community for open-source coding tools will include:
1) scalability of analysis (something not realized in spreadsheets),
2) error reduction,
3) increased development effort due to the crowdsourcing nature of open-source tools,
4) a low-cost path from current deterministic analysis to a more robust probabilistic analysis approach.
\item \textbf{Digitizing geotechnical data collection:} Machine learning involves digitizing the natural (and often subconscious) ability humans have for recognizing patterns and drawing inferences from them.
Geotechnical engineering is one of the more imprecise engineering fields due to the inherent variability of geologic materials.
As a result, ``engineering judgment'', based largely on experience and intuition, has a significant role to play in practice.
For example, an experienced geotechnical engineer may be able to estimate grain size distribution, plasticity, and compactibility with only a quick glance at or touch of a soil sample.
With the ubiquity of high-resolution cameras, a machine learning algorithm for predicting soil properties from photos seems feasible and valuable.
This example is one of many that I believe are overdue for exploration in geotechnical data collection.
I hope to use my experience in industry to identify areas where this type of research will provide a positive impact on the efficiency and reliability of geotechnical data collection.
\end{enumerate}
\end{enumerate}

\section{Research Support}
\subsection*{Funding Sources}
\begin{enumerate}
\item \textbf{Royalty Research Fund} (UW internal funding)\\ 
Status: Funded (June 2022)\\
Amount: \$38k\\
Duration: 1 year
\item \textbf{NSF Engineering Research Initiation (ERI)}\\
Status: Application in progress (due October 11, 2022)\\
Amount: \$200k\\
Duration: 2 years
\item \textbf{KEEN CURE}\\
Status: Tentative (pending discussion with Jeff W.)\\
Amount: \$12k\\
Duration: 6 months
\item \textbf{NSF Faculty Early Career Development (CAREER)}\\
Status: First planned submission July 2024\\
Amount: \$500k\\
Duration: 5 years

\end{enumerate}
\subsubsection*{Other potential funding sources:}

\textbf{NSF RAPID grant:} These grants are awarded without panel review for collection of perishable data. Successful examples include instrumentation of rare planned events (e.g., rock blasting for a highway project). 
This grant could be a good candidate for collaboration with WSDOT by using a large infrastructure project activity as an opportunity to collect a research dataset.
\\
\\
\noindent \textbf{Unknown:} Some of my research interests involve tool development and data collection methods.
These are more applied and less fundamental than what the NSF typically funds, so I need to research alternative funding sources. 
WSDOT and the FHWA are the agencies I plan to look into first.
Additionally, some of my research ideas involving digitization of data collection may be well-suited for commercial development. 
Although I'm not personally interested in starting a company, collaboration with a local incubator could provide some interesting opportunities for myself and some motivated students.

\subsection*{Student Involvement}
I plan to involve undergraduate UW Tacoma students in my research through summer research experiences, research credits, and paid hourly research tasks.
I hope to recruit student researchers through in-class introductions to research in TCE 367 and 436 (see \textbf{Integration of Teaching and Service}).

Through my connections with UW Seattle and Montana Tech, I plan to have access to graduate student researchers who can provide higher level (and longer term) research for my projects.

\subsection*{Post-Doc Support}
I would like to get funding for a 2-year post-doc position. 
This would allow me to get a higher level of support from a self-guided person, and to gain mentoring experience at a higher level.
I will need to research funding opportunities.

\section{Publication Plans}
I plan to publish two peer-reviewed papers per year (on average), with at least one of those being a journal article.
Currently, a significant issue with this publication plan is that it focuses exclusively on one subset (1a -- seismically-induced rock slope failure) of my research topics.
This is something I hope to change over time as I develop clearer research plans for other areas. 
Although the following section is outlined for approximately 5 years, I hope to replace 25 to 50\% of the publications listed below with publications from other research areas as I make progress in them.
\\
\\
\noindent \textbf{Journal Articles:} My strategy is to use feedback from my conference papers and new data from my research to develop more detailed papers for journal publication. 
I will target the highest impact journal that I think is a good fit for my paper, and pursue lower impact journals if I'm not able to get into my target journals.
\\
\\
\noindent Planned journal articles:
\begin{itemize}
\item JA1 - ``Seismically induced failure in homogeneous rock slopes''
\\Target Journal 1: Engineering Geology; Target Journal 2: Journal of Geophysical Research -- Earth Surface
\\ Authorship: Primary
\item JA2 - ``Dynamic boundary conditions and wave propagation in particulate discrete element models''
\\Target Journal: Numerical and Analytical Methods in Geomechanics
\\ Authorship: Primary
\item JA3 - ``A sliding block correlation for predicting seismically induced rock slope failure''
\\ Target Journal: Soil Dynamics and Earthquake Engineering
\\ Authorship: Primary
\item JA4 - ``Seismically induced failure in rock slopes with pre-existing discontinuities''
\\ Target Journal 1: Journal of Geophysical Research -- Earth Surface; Target Journal 2: Engineering Geology
\\ Authorship: Primary
\item JA5 - ``A seismic risk analysis framework for regional-scale rock slope stability''
\\ Target Journal: Earthquake Spectra
\\ Authorship: Primary
\end{itemize}

\noindent \textbf{Conference Papers:} My strategy for conference papers is to target publication at conferences I plan to attend in person and that will provide networking opportunities for funding and research collaboration.
My goal is to write papers of interest to the engineering community that show preliminary results in a limited way that allows the paper to be developed more fully for future journal publication.
\\
\\
\noindent Planned conference papers:
\begin{itemize}
\item CP1 - ``Estimating seismically induced rock slope failure volume using a sliding block correlation''
\\Conference: GeoCongress (2023)
\\ Authorship: Primary
\item CP2 - ``Modeling Rock Fracture with PFC3D Distinct Element Software''
\\Conference: American Geophysical Union (2023)
\\ Authorship: Secondary
\item CP3 - ``Failure mechanisms and wave dynamics in rock slopes''
\\ Conference: 8th International Conference on Earthquake Geotechnical Engineering (2024)
\\ Authorship: Primary
\item CP4 - ``Scalability of simplified methods
\\ Conference: National Earthquake Conference (2025)
\\ Authorship: Primary
\item CP5 - unknown
\\ Conference: 9th International Conference on Earthquake Geotechnical Engineering (2028)
\\ Authorship: Primary
\end{itemize}

%\noindent \textbf{Datasets:} An increasing and important trend in academia involves publishing datasets.
%For NSF-funded research in particular, a well-organized, published data set is almost required.
%Published datasets can be cited using a digital object identifier (doi) and are increasingly being considered as evidence of scholarship for tenure-track faculty in large research institutions.
%I plan to publish my research datasets on the NSF-hosted DesignSafe repository.

\section{Integration of Teaching and Service}
To keep momentum on my research efforts as I balance other responsibilities, I plan to integrate research into my teaching and service activities.
\subsection*{Teaching integration}
My interest in digitization of geotechnical data collection will lend itself to integration in TCE 367 (Intro to geotechnical engineering).
I plan to prepare a lab module where students will use photos of soil samples and the grain-size distribution lab results that they generate in the course to look for correlations between the photos and the lab data.
This exercise will expose students to research, teach them about a practical aspect of geotechnical engineering, and provide a basis for some of my future research.
I also hope that through this in-class research exposure, I'll be able to recruit one or two participants to continue the research with me as part of a summer research experience.

I don't have as clear of an idea for integrating research into TCE 436 (Foundation design), but I hope to have a research-focused module in that class, too.
The most likely candidate is probably something to do with the coding tools I hope to develop.
One of the benefits of coding tools is the ability to perform probabilistic analysis.
I may be able to develop a module that compares the understanding of a foundation design with respect to probabilistic vs. deterministic analysis under different amounts of subsurface variability.
This kind of comparison may help inform future publication of guidance on the conditions that warrant probabilistic analysis in geotechnical engineering.

\subsection*{Service integration}
One of my service interests is STEM outreach to pre-college students, particularly those from underrepresented groups.
I hope to integrate research into this service interest by having the undergraduate students I supervise present some of their work to local high school students.
This kind of activity will give undergraduate students presentation experience, provide exposure to civil engineering for high school students, and provide me with a concrete example of ``broader impacts'' activities.
``Broader Impacts'' (BIs) are a major component to the evaluation of research proposal quality by the NSF.
For the NSF CAREER grant in particular, BIs need to be directly related to the research project proposed.

\section{Timeline}
The table below presents my proposed timeline for research proposals and publications.
As mentioned in \textbf{Publication Plans}, I hope that this schedule will change to include proposals and publications in support of more than just one of my research interests (1a -- seismically-induced rock slope failure).
\begin{table}[h!]
\begin{threeparttable}
\begin{center}
\begin{tabularx}{1.0\linewidth}{|l|X|X|X|X|X|X|}
\hline
     &\textbf{2022} &\textbf{2023} &\textbf{2024} &\textbf{2025} &\textbf{2026} &\textbf{2027} \\ \hline
     \textbf{Jan} &\C{2022} &\C{2023} &\C{2024} &\C{2025} (\$ $\Rightarrow$: CAREER) &\C{2026} &\C{2027} \\ \hline
     \textbf{Feb} &\C{2022} &\C{2023} &\C{2024} T:JA3 &\C{2025} T:JA4&\C{2026} &\C{2027} \\ \hline
     \textbf{Mar} &\C{2022} &\C{2023} &\C{2024} &\C{2025} &\C{2026} &\C{2027} \\ \hline
     \textbf{Apr} &\C{2022} &\C{2023} S: 2-yr review, T:JA2 &\C{2024} &\C{2025} &\C{2026} S: tenure &\C{2027} \\ \hline
     \textbf{May} &\C{2022} &\C{2023} &\C{2024} &\C{2025} &\C{2026} &\C{2027} \\ \hline
     \textbf{Jun} &\C{2022} &\C{2023} (\$ $\Rightarrow$: ERI), T:CP3 &\C{2024} &\C{2025} (\$ $\otimes$: ERI)&\C{2026} &\C{2027} \\ \hline
     \textbf{Jul} &\C{2022} \$ $\Rightarrow$: RRF &\C{2023} \$ $\otimes$: RRF &\C{2024} S: CAREER &\C{2025} &\C{2026} &\C{2027} \\ \hline
     \textbf{Aug} &\C{2022} &\C{2023} &\C{2024} &\C{2025} &\C{2026} &\C{2027} T:CP5\\ \hline
     \textbf{Sep} &\C{2022} S: ERI, T:CP1 &\C{2023} \$ $\Rightarrow$: KEEN, \newline \$ $\otimes$: KEEN  &\C{2024} T:CP4&\C{2025} T:JA5&\C{2026} &\C{2027} \\ \hline
     \textbf{Oct} &\C{2022} &\C{2023} &\C{2024} &\C{2025} &\C{2026} &\C{2027} \\ \hline
     \textbf{Nov} &\C{2022} T:JA1&\C{2023} &\C{2024} &\C{2025} &\C{2026} &\C{2027} \\ \hline
     \textbf{Dec} &\C{2022} T:CP2 &\C{2023} &\C{2024} &\C{2025} &\C{2026} &\C{2027} \\ \hline
\end{tabularx}

\begin{tablenotes}
    \small
    \item S: Submission due
    \item T: Target submission
%    \item P: Manuscript publication
    \item \$ $\Rightarrow$: Start of funding 
    \item \$ $\otimes$: End of funding
    \item (Items in parentheses are pending award)
\end{tablenotes}
\end{center}
\end{threeparttable}
\end{table}

\noindent \textbf{Note:} I also hope to participate in at least one Geotechnical Extreme Event Reconnaissance (GEER) mission in the next 5 years.
Since these missions occur in response to natural disasters, they cannot really be planned in advance.
GEER missions generally result in a data report and peer-reviewed journal articles.
As a member of GEER, and having taken a workshop in the Natural Hazard Emergency Response Infrastructure (NHERI) RAPID center this summer, I will be on a list of qualified geotechnical engineers for deployment and data collection following natural disasters.
\end{document}